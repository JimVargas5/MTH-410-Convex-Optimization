\documentclass[12]{article}

\usepackage[margin=1in]{geometry}
\usepackage{setspace}
\usepackage{graphicx}
\usepackage{amsmath}
\usepackage{mathrsfs}
\usepackage{amsfonts}
\usepackage{tikz}
\usepackage{pgfplots}
\usepackage{verbatim}

\DeclareMathOperator{\tr}{tr}
\newcommand{\Lagr}{\mathscr{L}}

\begin{document}
\noindent
\textbf{Part 2} Jim Vargas MTH 410 HW4 \\

	Let it be assumed that I will be using the method of Lagrange multipliers for all of the following problems.\\
	
\noindent
\textbf{1) a)}

	Given $f(x)=x_1^2+2x_1 x_2+3x_2^2+4x_1+5x_2+6x_3$ for $x\in \mathbb{R}^3$, the gradient and Hessian of $f$ are, respectively,
	$$\displaystyle{
	\nabla f(x)=\begin{pmatrix}
	2x_1+2x_2+4 \\
	2x_1+6x_2+5\\
	6
	\end{pmatrix}
	}$$
	$$\displaystyle{
	\nabla^2 f(x)=\begin{pmatrix}
	2 & 2 & 0 \\
	2 & 6 & 0 \\
	0 & 0 & 0  
	\end{pmatrix}.
	}$$
It follows that $\det{\nabla^2 f(x)}=0$ and $\tr{\nabla^2 f(x)}=8$, and so $f$ is convex. \\

\noindent
\textbf{1) b)}

	The Lagrangian function for the problem is 
	$$\displaystyle{
	\Lagr (x,\lambda)= f(x)-{\lambda}_1 (x_1+2x_2-3) -{\lambda}_2 (4x_1+5x_3-6)
	}$$
with $\lambda=({\lambda}_1,{\lambda}_2)$. Differentiating component-wise with respect to $x$ and setting each to zero to find a unique minimizer $x$, in conjunction with the complimentary slackness conditions, we get the following equations:
\begin{align*}
0&=2x_1+2x_2-{\lambda}_1-4{\lambda}_2+4 \\
0&=2x_1+6x_2-2{\lambda}_1+5 \\
0&=-5\lambda_2+6 \\
0&=\lambda_1(x_1+2x_2-3) \\
0&=\lambda_2(4x_1+5x_3-6).
\end{align*}
The third equation yields $\displaystyle{\lambda_2=\frac{6}{5}}$. After substituting this and doing some manipulations, we get $\displaystyle x_1=\frac{\lambda_1}{4}+\frac{37}{20}$ and $\displaystyle x_2=\frac{\lambda_1}{4}-\frac{29}{20}$ and $\displaystyle x_3=\frac{-\lambda_1}{5}-\frac{7}{24}$. Eventually more computation yields $\displaystyle \lambda_1=\frac{27}{5}$ or $\lambda_1=0$, but $\lambda_1=0$ does not satisfy the constraints. Finally, the optimal solution is $\displaystyle x=\left(\frac{16}{5},\frac{-1}{10},\frac{-34}{24}\right)$ with $\displaystyle f(x)=\frac{1377}{100}$.\\


\noindent
\textbf{2)}
	
	The Lagrangian function for this problem can be written as 
	$$\displaystyle
	\Lagr (x,\lambda)=\frac{1}{2}\left\langle Q^\top x,x\right\rangle -c^\top x+d-\langle \lambda A^\top,x \rangle+\langle \lambda,b \rangle.
	$$
Taking the gradient of $\Lagr$ with respect to $x$ and setting it to zero, we get
\begin{align*}
0&=Q^\top x - c^\top-\lambda A^\top \\
\Rightarrow x&=Q^{-1} (c^\top +\lambda A^\top).
\end{align*}
With the domain restriction, this means (at least, I think) 
\begin{align*}
b&=A(Q^{-1} (c^\top +\lambda A^\top)) \\
\Rightarrow \lambda &= (AQ^{-1}A^\top)^{-1}(b-AQ^{-1}c^\top) \\
\Rightarrow x &= Q^{-1} (c^\top +[(AQ^{-1}A^\top)^{-1}(b-AQ^{-1}c^\top)] A^\top).
\end{align*}
\pagebreak

\noindent
\textbf{3)}

	The Lagrangian function for this problem is 
	$$\displaystyle
	\Lagr (x,y,\lambda)=x+2y-\lambda (x^2+y^2-1).
	$$
Differentiating with respect to $x$ and $y$ and setting to zero, along with the complimentary slackness condition, yields
\begin{align*}
0&=1-2\lambda x \\
0&=2-2\lambda y \\
0&=\lambda (x^2+y^2-1).
\end{align*}
With these, we get $\displaystyle x=\frac{1}{2\lambda}$ and $\displaystyle y=\frac{1}{\lambda}$ with $\displaystyle \lambda^2=\frac{5}{4}$. Taking the positive root of $\lambda$ to maximize $f(x,y)=x+2y$, the solution is $\displaystyle (x,y)=\left(\frac{1}{\sqrt{5}},\frac{2}{\sqrt{5}}\right)$ with $f(x,y)=\sqrt{5}$.\\

\noindent
\textbf{4)}
	
	The Lagrangian function for this problem is 
	$$\displaystyle 
	\Lagr (x,\lambda)=x_1^2+x_2^2-\lambda_1 (-x_1) -\lambda_2 (-x_2) -\lambda_3 (5-x_1-x_2)
	$$
for $x\in \mathbb{R}^2,\, \lambda=(\lambda_1,\lambda_2,\lambda_3)$. Taking the derivative component-wise and setting them to zero, after some computations we get $\displaystyle x_1=\frac{-\lambda_1 -\lambda_3}{2}$ and $\displaystyle x_2=\frac{-\lambda_2 -\lambda_3}{2}$ with the following conditions on $\lambda_1 ,\, \lambda_2 ,\, \lambda_3$:
\begin{align*}
0&=\lambda_1 \left(\frac{-\lambda_1 -\lambda_3}{2}\right) \\
0&=\lambda_2 \left(\frac{-\lambda_2 -\lambda_3}{2}\right) \\
0&=\lambda_3 \left(5-\frac{-\lambda_1 -\lambda_3}{2}-\frac{-\lambda_2 -\lambda_3}{2}\right).
\end{align*}
There are a few possible combinations of the lambdas which satisfy the above system alone, however, simple methods will show that only $\lambda_3=-5,\,\lambda_1=\lambda_2=0$ will satisfy the given boundary conditions. Therefore the solution is $\displaystyle x=\left(\frac{5}{2},\frac{5}{2}\right)$ with $\displaystyle f(x)=\frac{25}{2}$.
\end{document}
